\documentclass{book}

\usepackage{graphicx}
\graphicspath{ {images/} }

\usepackage[T1]{fontenc}
\usepackage[utf8]{inputenc}
\usepackage{natbib}

% Para símbolos matemáticos
\usepackage{amsfonts}
\usepackage{amsmath}
\newcommand{\Mod}[1]{\ (\mathrm{mod}\ #1)}

% Símbolo de euro
\usepackage[gen]{eurosym}

% Para meter código
\usepackage{listings}
% https://www.overleaf.com/learn/latex/Code_Highlighting_with_minted
\usepackage{minted}
\usemintedstyle{borland}

% Para \url{}
\usepackage{hyperref}
\hypersetup{
    colorlinks=true,
    linkcolor=blue,
    filecolor=magenta,      
    urlcolor=cyan,
}

\title{Estudio de viabilidad del uso de librerías de criptografía homomórfica en procesos reales}
\author{Javier Junquera Sánchez}
\date{Septiembre 2019}

\begin{document}

  \maketitle

  \chapter*{Resumen}
\label{chap:resumen}

% Máximo 200 palabras
Se conoce como criptografía homomórfica al conjunto de técnicas destinadas a cifrar los datos de tal forma que las operaciones que se apliquen sobre el texto cifrado se manifiesten en el texto plano al descifrar. En este trabajo analizaremos las distintas técnicas existentes para este propósito, cuáles son sus bases teóricas, y evaluaremos si es viable o no utilizar las implementaciones disponibles en sistemas destinados al uso en producción. 

Las principales implementaciones de criptografía homomórfica protegen el texto utilizando un esquema de cifrado llamado \textit{Learning With Errors}, y estarán categorizadas como \textit{Partially Homomorphic Encryption}, \textit{Somewhat Homomorphic Encryption} (SHE) y \textit{Fully Homomorphic Encryption} (FHE) en función de las propiedades homomórficas que cumplan. Para nuestra evaluación construiremos un sistema que utilice una implementación de tipo SHE y otra de tipo FHE, y compararemos los resultados desde el punto de vista de la eficiencia, de la capacidad de cómputo y de la facilidad de desarrollo. 

Concluiremos mostrando cómo existe cierta viabilidad desde el punto de vista tecnológico a la hora de utilizar sistemas de criptografía homomórfica, pero la viabilidad económica dependerá tanto del valor del activo como del riesgo que se busque mitigar, pues todavía no existe una solución universal.
  \chapter*{Abstract}
\label{chap:abstract}

Homomorphic Encryption is the set of techniques designed to encrypt data so that an operation performed over encrypted data remains in the data when decryption. In this work, we will analyze the different existing techniques for this purpose, what are its theoretical bases, and we will evaluate the viability for using the different existing implementations in production systems.

The main homomorphic encryption implementations protect the text using a schema known as \textit{Learning With Errors} and are categorized as \textit{Partially Homomorphic Encryption}, \textit{Somewhat Homomorphic Encryption} (SHE) and \textit{Fully Homomorphic Encryption} (FHE) depending on the homomorphic encryption properties they fulfill. For our implementation, we will build a system using both a SHE, and an FHE implementation, and we will compare the results from the point of view of the efficiency, the computing capacity and the easiness of development.

We will conclude showing how it exists certain viability from the point of view of the technology when using homomorphic encryption systems, but the economic viability depends on both the active value and the risk to be mitigated, as there is still no a universal solution.
  \chapter*{Agradecimientos}
\label{chap:agradecimientos}

  \tableofcontents{}
  \listoftables{}
  \listoffigures{}

  \chapter{Introducción}
\label{chap:intro}

Para todo $ A,B \in \chi{} $, una operación $ \bigoplus $, y una función $f$; si se cumple $ f(A) \bigoplus f(B) = f(A \bigoplus B) $, $ f $ es una función homomórfica con respecto a $ \bigoplus $ en $ \chi{} $. Cuando una función criptográfica cumple esta condición con alguna operación se dice que es maleable (\cite{dolev_non-malleable_1991}).

Aunque es una propiedad que podría no ser deseable en muchos ámbitos (por ejemplo, en un escenario en el que además de confidencialidad se requiera integridad, permitiría a un atacante modificar los datos), si cumple ciertas condiciones puede tener numerosas aplicaciones. Así se crea el campo de estudio de la criptografía homomórfica.

Se conoce como criptografía homomórfica al conjunto de técnicas criptográficas destinadas a permitir operar con datos cifrados y que dichas operaciones se materialicen correctamente sobre los datos al descifrarlos. En función (principalmente) de las operaciones con las que se cumple esta premisa, o el sistema utilizado para procesar los datos antes y después de operar, los esquemas con propiedades homomórficas se categorizarán de una forma u otra. Aunque pueda haber muchas variantes de cada una de estas propiedades, la comunidad científica ha establecido criterios y notaciones para su estudio.

\section{Estandarización}

El consorcio "Homomorphic Encryption Standardization" (\cite{albrecht_homomorphic_2018}) ha ido desarrollando un estándar a lo largo de los años atendiendo a los avances en las distintas tecnologías que componen la criptografía homomórfica, y prestando un interés especial en las implementaciones necesarias para ponerla en práctica. Así, se han ido sucediendo las tres generaciones de criptografía homomórfica.

El último encuentro de trabajo del grupo se produjo el 17 de Agosto en Santa Clara, y en él se trabajará principalmente la eficiencia, la seguridad y la usabilidad de las librerías.

La documentación del estándar está dividida en tres \textit{white papers} y una lista de implementaciones conocidas.

\subsection{Seguridad}

En el documento (\cite{chase_security_2017}) analizan qué principios seguir para implementar esquemas de criptografía homomórfica y qué beneficios tienen estos esquemas para la seguridad de la información. También hacen un análisis de los ataques existentes a estos esquemas y qué parámetros matemáticos son los ideales para hacer estos esquemas lo más resistentes posible tanto en el panorama actual como en un escenario \textit{post-quantum}.

\subsection{Aplicaciones}

En el estudio (\cite{archer_applications_2017}) abordan para qué campos es útil la criptografía homomórfica. Mientras que se han estado centrando los esfuerzos en poder almacenar información en la nube de forma segura (cifrándola) se ha descuidado la seguridad de dicha información cuando sube a la nube para ser procesada. La criptografía homomórfica puede ser la solución a este problema, y en campos como:

\begin{itemize}
    \item La computación distribuida
    \item La protección de datos médicos que tengan que ser procesados en equipos potentes, ajenos a la institución médica
    \item La consulta de información de forma anónima: por ejemplo, una consulta DNS sin revelar a qué se está accediendo
\end{itemize}

En la última reunión del standard presentaron ejemplos de protocolos de distribución de datos y claves (\cite{troncoso-pastoriza_homomorphic_nodate}) utilizando nuevas implementaciones como Lattigo (\cite{noauthor_lattigo_2019}).

\subsection{API}

Por último el consorcio de estandarización busca establecer un modelo de almacenamiento común tanto de los datos como de las claves, y una terminología (lo llaman \textit{lenguaje ensamblador}) que sirva de lenguaje común para transmitir las ideas y los elementos mínimos que debe tener cualquier sistema de criptografía homomórfica. Están definidos en el documento (\cite{brenner_standard_2017}), y como veremos más adelante, encajan perfectamente con los elementos de las implementaciones. Son los siguientes:

\begin{itemize}
    \item \verb|SecKeygen|, \verb|PubKeygen|

    Tiene que haber un método de generación de clave privada (o secreta) y pública.

    \item \verb|SecEncrypt|, \verb|PubEncrypt|

    Define la posibilidad de que haya además de un sistema de cifrado público (usando la clave pública) que en determinados esquemas se pueda cifrar la información directamente con la clave privada.

    \item \verb|Decrypt|

    Tiene que haber un sistema para restaurar la información desde un texto cifrado.

    \item \verb|Eval|

    La llamada \verb|Eval| será el paraguas que recoja todas las operaciones homomórficas que se puedan realizar sobre el texto cifrado para que luego se materialicen en el descifrado.

\end{itemize}

Algunos esquemas introducirán otras herramientas enfocadas a procesar los datos, pero estarían en niveles superiores de la arquitectura, y son exclusivas de cada implementación.

\section{Objetivo del trabajo}

El objetivo de este trabajo es evaluar si es viable o no utilizar las tecnologías existentes en sistemas y procesos reales. Estudiaremos qué implementaciones hay, en qué consisten, y cuales serán las más idóneas (las más avanzadas, que puedan servir de muestra para inferir la viabilidad de las demás) atendiendo a:

\begin{itemize}
    \item La facilidad de uso: A fin de cuentas la documentación existente y la mayor o menor facilidad de uso se traduce en horas de salario de trabajadores altamente cualificados.
    \item Las capacidades de la tecnología: O qué problemas pueden resolverse con ella
    \item La eficiencia de la solución: Estudiar los tiempos de ejecución de las operaciones
\end{itemize}

  \chapter{Estudio teórico}
\label{chap:teoria}

Para la computación de criptografía homomórfica necesitaremos comprender las bases teóricas matemáticas y los distintos niveles de homomorfismo, además de las herramientas de notación y computación definidas por el estándar...

\section{Tipos}
\label{tag:tipos}

Las generaciones publicadas por el estándar guardan una estrecha relación con las capacidades que tienen los esquemas para trabajar con los datos cifrados. Estas capacidades que van desde la posibilidad de aplicar algún homomorfismo a poder trabajar libremente con el texto cifrado están categorizadas en tres niveles. Para comenzar, veremos cuales son estos tres tipos:

\begin{itemize}
  \item Partially Homomorphic Encryption

  Existe algún homomorfismo dentro del esquema de cifrado, pero este no es explotable para realizar cómputos arbitrarios con la información. Por ejemplo, el producto en RSA (ver \ref{form:rsa_product})

  \item Somewhat Homomorphic Encryption (SHE)

  Se pueden realizar operaciones arbitrarias, pero el sistema hace que a medida que se procesa la información aumenta el nivel de error del resultado (como veremos, los esquemas de cifrado son semi-probabilísticos), hasta destruir la información. Hay técnicas para aumentar este umbral de error, pero sigue teniendo límites.

  \item Fully Homormorphic Encryption (FHE)

  Los sistemas FHE son los más codiciados dentro del campo, porque permiten realizar cualquier cómputo con la información cifrada sin que aumente el nivel de error y se vuelva irrecuperable. Es cierto que actualmente estos esquemas actualmente son menos eficientes que los anteriores, pero como dicen en TFHE (\cite{gama_tfhe:_nodate}): "Si Spiderman puede balancearse sobre su cuerda el tiempo suficiente para lanzar una nueva cuerda, ¡puede volar!"

\end{itemize}


\section{Primitivas matemáticas}

La seguridad de los sistemas criptográficos se basa en problemas matemáticos que, si bien pueden ser resolubles, dicha resolución no es computacionalmente viable en un tiempo razonable: factorización de enteros, logaritmo discreto, ordenación de conjuntos...

La base de los principales sistemas de criptografía homomórfica modernos son problemas relacionados con unas estructuras algebraicas conocidas como retículos.

\subsection{Lattice-based encryption}

Un retículo o red (también conocidos como lattice en inglés) es un conjuntos de elementos similar a un espacio vectorial discreto generado por la combinación de una base vectorial concreta.

Por ejemplo, a los vectores $u = (2, 0)$ y $v = (-1, 3)$ serían la base de la red bidimensional generada por todas sus combinaciones $n*u + m*v$ (ver \ref{fig:lattice1}).

\begin{figure}[h]
  \caption{Espacio vectorial generado por u y v}
  \label{fig:lattice1}
  \includegraphics[]{lattice1}
\end{figure}

Es decir, el conjunto de puntos del gráfico \ref{fig:lattice1} sería un retículo, cuya base está formada por los vectores $u$ y $v$.

En función de lo cerca o lejos que estén los vectores que forman la base del origen se dirá que estas bases son largas o cortas\cite{wickr_what_2018}. Por ejemplo, el espacio anterior podría formarse con el mismo vector $v$, y con otro vector $w = (1, 0)$ más corto que $u$.

En la búsqueda de soluciones criptográficas resistentes a la computación cuántica se han encontrado útiles los siguientes problemas de retículos:

\begin{itemize}
  \item Short Vector Problem (SVP)

  Consiste en dada una base larga, buscar un vector corto lo más cercano al origen, sin ser el origen mismo (ver \ref{fig:svp}).

  \begin{figure}[h]
    \caption{Short vector problem \cite{wikipedia_contributors._lattice_2019}}
    \label{fig:svp}
    \includegraphics[]{svp}
  \end{figure}

  \item Short Basis Problem (SBP)

  Dada una base larga buscar una base corta del mismo espacio.

  \item Closest Vector Problem (CVP)

  Dada una base larga y un punto $P$ de la red, buscar el vector de la red formada por la base más cercano a $P$ (ver \ref{fig:cvp}).

  \begin{figure}[h]
    \caption{Closest vector problem \cite{wikipedia_contributors._lattice_2019}}
    \label{fig:cvp}
    \includegraphics[]{cvp}
  \end{figure}


\end{itemize}

Aunque estos problemas puedan parecer triviales a simple vista, su complejidad computacional aumenta exponencialmente con el aumento del número de dimensiones hasta hacerlo computacionalmente irresoluble.

Llamaremos Lattice-based encryption (para ajustarnos a la bibliografía, que está en su práctica totalidad escrita en inglés) a la aplicación de este conjunto de problemas a la criptografía.

En cuanto a nuestro caso, el problema del aprendizaje con errores aplicado a retículos será el núcleo de la criptografía homomórfica.

\subsection{Learn With Errors (LWE)}

El problema del aprendizaje con errores plantea la dificultad de determinar los componentes de una función en base a sus resultados cuando estos contienen errores\cite{apon_intro_nodate}.

Dada una base modular $q$, n vectores $a_i \leftarrow \mathbbb{Z}^m_q$, un vector $s \leftarrow \mathbbb{Z}^n_q$ y $n$ vectores $ e_i \leftarrow \chi{}^m $ tomados de una distribución de error (distribución de Gauss\cite{wikipedia_contributors._generalized_2019}) \chi{} \subset{} \mathbbb{Z}:

\begin{gather}
    \label{form:gen_lwe}
    b_i = (a_i \times s + e_i) \Mod{q}
\end{gather}

Conociendo $m$ pares $(a_i, b_i)$ no se puede determinar el valor $s$, y no se pueden diferenciar dichos pares de una distribución aleatoria\cite{t._zijlstra_learning_nodate}.

\subsubsection{Uso en criptografía}

En su estudio, Regev\cite{regev_learning_2010} formula un sistema de clave pública utilizando este problema:

\begin{enumerate}
  \item Generación de clave Pública

  Se emite como clave pública una colección de valores $(ai, bi)$ generada como hemos visto en \ref{form:gen_lwe}. El conjunto de vectores $a_i$ se interpretará en algunas implementaciones como una matriz $A \leftarrow \mathbbb{Z}_q^{m \times n}$.

    \begin{figure}[h]
      \caption{LWE \cite{halevi_homomorphic_2017}}
      \label{fig:lwe}
      \includegraphics[width=\textwidth]{lwe}
    \end{figure}

  \item Cifrado

  Para cada bit $x$ a cifrar se elige una de las parejas y se realiza la siguiente operación:

    \begin{gather*}
        \label{form:cifrado_lwe}
        (c1, c2) = (\sum_{j=1}^{m} a_i_j, \sum_{j=1}^{m} b_i_j + x * (q/2))
    \end{gather*}

  \item Descifrado

  El resultado realmente no es determinista, es decir, no devuelve realmente el valor cifrado. Para determinar el valor calcularíamos:

    \begin{gather*}
        \label{form:descifrado_lwe}
        x \approx c_2 - (c_1*s) \\
        b + x*(q/2) - a*s \\
        a*s + e + x*(q/2) - a*s \\
        e + x*(q/2) \approx x * (q/2)
    \end{gather*}

  Si se ha introducido $s$ correctamente, y siendo $e$ despreciable en comparación con $ q/2 $, obtendríamos como resultado $ x * (q/2) $, por lo que sabremos que $ x = 1$ si el resultado es cercano a $ q/2 $ y  $ x = 0 $ si el resultado es cercano a $ 0 $.

\end{enumerate}

En el apéndice \ref{appendix:test_lwe.py} puede verse un ejemplo de implementación del algoritmo, y el resultado de su ejecución.

\section{Generaciones}

Las distintas generaciones de esquemas de criptografía homomórfica se han ido cerrando en base a los encuentros de estandarización. Además de la propia estandarización, estos encuentros (concretamente, de la segunda generación en adelante) han servido para crear los grupos de trabajo que han hecho germinar los avances de la siguiente. Aunque la única categorización realmente aplicable a los esquemas es la que hemos visto al principio de este capítulo (\ref{tag:tipos}), es interesante ver las distintas fases para conocer la evolución, pues normalmente la diferencia entre unas y otras ha sido símplemente el desarrollo en profundidad de una idea; o el mecanismo de una ha surgido mediante la implementación de una idea radicalmente opuesta a la base teórica de la anterior.


Además las tecnologías de, por ejemplo, la segunda generación, no han sido sustituidas por las de la tercera, si no que este salto generacional indica la existencia de un modelo más maduro hacia la consecución del verdadero objetivo: sistemas FHE eficientes. Siguiendo con el ejemplo, los esquemas de la segunda generación se siguen utilizando para cálculos acotados que requieren cierta eficiencia. Este es el motivo de que, como veremos más adelante, hayamos elegido una tecnología de cada una de estas generaciones para desarrollar nuestra implementación.

\subsection{Pre-HE}

Dentro de la categoría de esquemas previos a la criptografía homomórfica se encuentran aquellos que, o bien tienen propiedades homomórficas de forma casual, o bien no cumplen las condiciones necesarias para que se puedan utilizar en ningún sistema. Hablaremos de RSA por lo intuitivo que es para comprender la criptografía homomórfica, y del sistema desarrollado por Boneh, Goh y Nissim por ser uno de los primeros orientados correctamente a la computación con criptografía homomórfica.

\begin{itemize}

    \item RSA

    Las propiedades matemáticas de RSA lo convierten en un esquema con homomorfismo en el producto. Para cifrar y descifrar, en RSA se exponencia el elemento al que se le desea aplicar la operación. Siendo $e$ la clave de cifrado y $d$ la clave de descifrado, la aplicación de RSA (puro) sobre el mensaje $m$ sería tal que:

    \begin{gather*}
        c = m^e \Mod{n} \\
        m = c^d \Mod{n}
    \end{gather*}

    Esto hace que, si multiplicamos dos mensajes cifrados:

    \begin{gather*}
        c_1 = m_1^e \Mod{n}, c_2 = m_2^e \Mod{n} \\
        c_1 * c_2 = m_1^e * m_2^e \Mod{n} = (m_1*m_2)^e \Mod{n}
    \end{gather*}

    Podamos obtener el producto de los dos elementos al descifrar:

    \begin{gather*}
        \label{form:rsa_product}
        (m_1^e * m_2^e)^d \Mod{n} \\
        ((m_1*m_2)^e)^d \Mod{n} \\
        m_1 * m_2
    \end{gather*}

    \item Criptosistema de Boneh–Goh–Nissim (\cite{hutchison_evaluating_2005})

    Permite evaluar circuitos lógicos en forma normal disyuntiva (reducido a puertas lógicas \verb|or|, \verb|and| y \verb|not|) sobre texto cifrado. Se traduce en la capacidad de evaluar polinomios de segundo grado. Aunque en comparación con los esquemas actuales este parezca "de juguete", es un gran aporte a la hora de impulsar la investigación en criptografía homomórfica.

\end{itemize}

\subsection{Primera generación}

\begin{itemize}

    \item Bootstrapping: Fully homomorphic encryption using ideal lattices

    La técnica de \textit{bootstrapping} de Gentry (\cite{gentry_fully_2009}) revoluciona la criptografía homomórfica. Estipula que para crear un esquema de cifrado que permita la evaluación arbitraria de circuitos lógicos símplemente hace falta un esquema de cifrado que pueda realizar la operación de descifrado sin realmente descifrar el texto. Esto se consigue mediante algo parecido a mezclar el texto cifrado (en el que se produce ruido tras hacer una operación) con una versión cifrada de la clave secreta (conocida como clave de evaluación) para que se elimine el ruido.

    Por ejemplo, cuando se evalúa usando la técnica de \textit{bootstrapping} la operación $ c_a + c_b$ con $c_a, c_b$ textos cifrados, $e(x)$ función de cifrado y  $d(x)$ función de descifrado, el resultado es $e(d(c_a + c_b) )$. En la operación de descifrado intermedia se elimina el ruido, y el propio esquema hace que el evaluador de la operación no pueda revelar la clave secreta. (\cite{noauthor_homomorphic_nodate-2}).

    El principal problema de la operación de \textit{bootstrapping} es que consume mucho tiempo. En la segunda generación todos los esfuerzos se centran en crear esquemas más rápidos.

\end{itemize}

\subsection{Segunda generación}

\begin{itemize}

    \item BGV

    Esquema de cifrado (enunciado como FHE, pero finalmente categorizado como SHE) basado en LWE que permite evitar el costoso método de bootstrapping mediante la introducción del concepto de "leveled homomorphic encryption": permite hacer esquemas más eficientes (no se va eliminando el error mediante boostrapping), pero el número de cómputos que se puede hacer sobre el texto cifrado está acotado hasta un punto en el que, el error acumulado (el error "despreciable" al que hacíamos referencia en \ref{form:descifrado_lwe}), hace irrecuperable el mensaje.

    % https://crypto.stackexchange.com/questions/15794/difference-between-leveled-fhe-and-normal-fhe-scheme

    https://eprint.iacr.org/2011/277

    \item BFV

    Es una implementación optimizada de BGV sobre RLWE (ring-LWE, sistema de LWE sobre elementos de un anillo algebraico, \cite{wikipedia_contributors._anillo_2019}) que mejora la eficiencia del esquema y aumenta la cota de cómputo (el número de operaciones que se pueden realizar hasta que el error del cifrado lo vuelva inconsistente) de una técnica conocida como realinearización.

    % TODO Escribir más de realinearización
    \if false
    The BFV scheme cannot perform arbitrary computations on encrypted data.
        Instead, each ciphertext has a specific quantity called the `invariant noise
        budget' -- or `noise budget' for short -- measured in bits. The noise budget
        in a freshly encrypted ciphertext (initial noise budget) is determined by
        the encryption parameters. Homomorphic operations consume the noise budget
        at a rate also determined by the encryption parameters. In BFV the two basic
        operations allowed on encrypted data are additions and multiplications, of
        which additions can generally be thought of as being nearly free in terms of
        noise budget consumption compared to multiplications.
    \fi

    https://eprint.iacr.org/2012/144

    % TODO A paper by Costache and Smart [CS16]gives some initial comparisons between BGV, BFV

    \item CKKS

    No está descrito en el estándar, pero sí se plantea su implementación y se extiende su uso en varias librerías. Su funcionamiento consiste en introducir una función de reescalado del texto cifrado a medida que se va trabajando con él. El reescalado trunca el mensaje cifrado (operación equivalente a dividir para reducir su magnitud) eliminando progresivamente el error cada vez que se opera. De esta forma se pueden realizar operaciones aritméticas con número reales (pertenecientes a \mathbbb{R}) e ir eliminando el error (se reduce cuando se trunca el mensaje) a medida que se va operando.

    Introduce además una técnica de codificación de los datos que nos permitirá trabajar con ellos como vectores, permitiendo aplicar técnicas SIMD (Single Instruction Multiple Data, \cite{wikipedia_contributors._simd_2017}) para paralelizar determinadas tareas, técnica que utilizaremos en nuestra implementación con Microsoft SEAL (\ref{tag:msfseal}).

    https://link.springer.com/chapter/10.1007%2F978-3-319-70694-8_15

\end{itemize}

\subsection{Tercera generación}

\begin{itemize}

  \item GSW

  El principal esquema de la tercera generación ya es considerado realmente FHE, pues permite la evaluación de todas las operaciones necesarias para poder realizar cualquier cómputo, todas las veces que hagan falta (sin cotas). En este nuevo esquema se implementa el problema LWE sobre matrices, tratando la clave como una matriz cuyo auto-vector es, aproximadamente (por el pequeño error del problema LWE), su autovector. En este esquema desaparece la necesidad de utilizar la clave de evaluación para operar (hace opcional la operación de boostrapping), pudiendo trabajar directamente con el dato cifrado.

  https://eprint.iacr.org/2013/340

  \item TFHE

  El éxito de GSW como equema FHE reside en que todos los parámetros son extremadamente pequeños comparados con $q$ (la base modular), y la implementación de mecanismos para que el error acumulado crezca muy lento. De esta forma, no es necesaria la operación de bootstrapping en la mayoría de los casos (que hasta el momento, tardaba alrededor de 6 minutos por operación (\cite{ducas_fhew:_2014})), y siempre puede aplicarse cuando el cálculo vaya a crecer mucho. El esquema FHEW logra un hito reduciendo el tiempo de bootstrapping hasta 1 segundo por operación, pero no es el último paso

  Basado en una aproximación a LWE conocida como TLWE (Thorus Learn With Errors) en la que los parámetros de LWE se definen sobre un el espacio de un toro \cite{cheon_faster_2016}, y siguiendo la trayectoria del esquema FHEW (\cite{ducas_fhew:_2014}), TFHE puede realizar la técnica de bootstrapping en menos de $0.1$ segundos. Esto permite su aplicación sobre GSW sin perder eficiencia. Además, reduce el tamaño de las claves hasta 16MB (en lugar del GB que ocupaban hasta el momento) manteniendo el mismo nivel de seguridad.

  Utilizaremos la librería homónima para construir parte de la implementación del trabajo.

  https://eprint.iacr.org/2016/870

\end{itemize}

  \chapter{Implementaciones}
\label{chap:libs}

\section{Librerías}

Hay varias implementaciones de criptografía homomórfica, y varias soluciones por cada una de las generaciones. Aunque la mayoría de las implementaciones están escritas para \verb|C/C++/C#| Recientemente han ido apareciendo algunas nuevas que buscan, principalmente, adaptar los esquemas de cifrado a otros lenguajes. Aunque esto es una nueva noticia, nosotros nos centraremos en las "clásicas" evaluadas por el consorcio de standarización:

\begin{itemize}
    \item Segunda generación
    \begin{itemize}
        \item HELib
        \item Microsoft SEAL
        \item PALISADE
        \item HeaAn
        \item LoL
        \item NFLlib
    \end{itemize}
    \item Tercera generación
    \begin{itemize}
        \item TFHE
        \item FHEW
        \item cuHE
    \end{itemize}
\end{itemize}

Para nuestro trabajo utilizaremos las librerías Microsoft SEAL (como representante de la segunda generación de criptografía homomórfica) y THFE (como representante de la tercera). Para la segunda generación habría sido también una muy buena opción utilizar PALISADE, pero la documentación es mucho menor, y no incluye el esquema CKKS (necesario para trabajar con números reales).

\section{Microsoft SEAL}
\label{tag:msfseal}

Esta librería de código abierto desarrollada por Microsoft (\textit{SEAL} de ahora en adelante) busca ofrecer una opción asequible para los desarrolladores de implementar soluciones con criptografía homomórfica.

Es muy fácil de instalar, no tiene dependencias externas, y está diseñada para ser construida en cualquier entorno con \verb|cmake|.

Cuenta con dos esquemas de cifrado de segunda generación (BGV y CKKS) y dentro de su código fuente se incluyen varios ejemplos que muestran cómo se opera con ella. En estos ejemplos, ordenados para conocer las distintas herramientas, muestran todo lo necesario para empezar a trabajar.

A la hora de implementar una idea en SEAL, hay dos aspectos clave a tener en cuenta:

\begin{enumerate}
  \item Elegir el esquema de cifrado que nos permita codificar todo correctamente
  \item Estudiar si la operación realmente es realizable con estos esquemas (recordemos que SHE tiene una cota de cómputo)
\end{enumerate}

Además, hay que ser consciente de que bajo determinados usos, puede ser insegura. Por ejemplo, no se debe permitir el descifrado desde un entorno no controlado (no es CCA seguro (\cite{peng_danger_2019})).

\subsection{API}

La API para operar con SEAL está codificada en los siguientes elementos:

\begin{itemize}
  \item EncryptionParameters

  Parámetros descriptivos de los elementos criptográficos

  \item SEALContext

  Contexto del texto cifrado (cambia a medida que se opera)

  \item Claves

  \begin{itemize}

    \item SecretKey

    Clave secreta para descifrar los datos

    \item PublicKey

    Clave pública para descifrar los datos

    \item RelinKeys

    Claves públicas para realinearizar los datos y reducir el nivel de error acumulado tras operar

    \item GaloisKeys

    Claves públicas para trabajar con rotaciones en los vectores cifrados (en nuestro ejemplo no las usamos)

  \end{itemize}

  \item Encryptor, Decryptor

  Sistemas para cifrar y descifrar los datos

  \item Evaluator

  Es el componente que realiza las operaciones sobre los datos cifrados

  \item Encoders

  Encargado de codificar los datos en función del esquema que se desee utilizar

\end{itemize}

A continuación explicaremos más detalladamente el funcionamiento de los componentes más complejos.

\subsection{EncryptionParameters}

Los parámetros criptográficos dependerán del esquema con el que se quiere trabajar.

\begin{itemize}

  \item BFV

  BFV es el esquema "básico" de SEAL. Permite trabajar con números enteros, y operar con ellos hasta que se alcanza el límite máximo de error. En BFV, este límite se podría conceptualizar como un cubo de fichas que se gastan cada vez que se opera.

  Hay algunas operaciones que son casi gratuitas (la suma y la resta), y la multiplicación es muy costosa. Una vez se vacía este cubo (codificado en el parámetro \verb|noise_budget|), nuestro cifrado quedará corrupto y no se puede recuperar el texto.

  Además de \verb|noise_budget|, hay tres parámetros configurables que guardan una estrecha relación:

  - \verb|poly_modulus_degree|

  Determina el módulo del polinomio usado para realizar las operaciones criptográficas (recordemos que implementa LWE sobre un anillo de polinomios, ver \ref{tag:bfv}). Se expresará como una potencia de 2 que cuanto más grande sea, más operaciones permitirá hacer, pero serán más lentas.

  - \verb|coeff_modulus|

  Es un vector de números primos que determina el módulo del texto cifrado. A mayor módulo, mayor será el \verb|noise_budget|. Cuando decíamos que a mayor \verb|poly_modulus_degree|, podíamos hacer más operaciones, era porque el número de bits de \verb|coeff_modulus| está acotado por el de \verb|poly_modulus_degree| de la forma indicada en la figura \ref{table:poly_vs_coeff_modulus}

    \begin{table}
        \centering
        \begin{subtable}
            \centering
            \begin{tabular}{  c  c  }
                \hline
                \verb|poly_modulus_degree|  & Número de bits de \verb|coeff_modulus| \\ [0.5ex]
                \hline
                \hline
                1024  & 27  \\
                2048  & 54  \\
                4096  & 109 \\
                8192  & 218 \\
                16384 & 438 \\
                32768 & 881 \\ [1ex]
                \hline
            \end{tabular}
      \end{subtable}
       \caption{Relación entre \textit{poly\_modulus\_degree} y y el número de bits de \textit{coeff\_modulus\_degree}}\label{table:poly_vs_coeff_modulus}

    \end{table}

  - \verb|plain_modulus|

  Modulo del texto plano. El consumo de \verb|noise_budget| se produce de forma logarítmica en base al tamaño de \verb|plain_modulus|, por lo que cuanto más pequeño es, más operaciones podremos realizar. También debe ser menor que \verb|poly_modulus_degree|.

  \item CKKS

  En la implementación del esquema CKKS podremos trabajar con número decimales, pero tendremos que realizar algunas gestiones adicionales a las de BGV para evitar que se pierda la integridad de nuestros cálculos.

  Comparte con BGV tanto \verb|poly_modulus_degree| como \verb|coeff_modulus|, pero desaparece el elemento \verb|plain_modulus| (la integridad del texto cifrado ya no se evaluará con el \verb|noise_budget|), y los valores elegidos para \verb|coeff_modulus| tendrán otras implicaciones.

  Aparece un elemento nuevo: la escala. De forma análoga a la que hemos implementado en la maqueta de THFE (ya lo veremos en \ref{chap:poc}) para codificar los números decimales se aplica una escala (se multiplican por un valor muy alto, potencia de 2) y se tratan como número enteros.

  Esta escala aumentará drásticamente cada vez que se multipliquen dos elementos. Siendo $N$ el tamaño del primer factor, y $M$ el del segundo, el resultado del producto tendrá como tamaño $M+N-1$. Para evitar que el número desborde su tamaño máximo, tras un producto puede reducirse la escala. La operación de reescalado implementada en SEAL trunca el valor del texto cifrado tantos bits como tenga el último elemento del vector \verb|coeff_modulus|, y elimina este elemento (cambiando el contexto de cifrado). De esta forma, siendo $P$ el tamaño del elemento eliminado de \verb|coeff_modulus| el texto cifrado pasaría a tener un tamaño $(M+N-1)/P$. Una vez se terminan los elementos de \verb|coeff_modulus| no se puede seguir operando, y se debe conservar siempre uno para poder realizar la operación de descifrado.

  Para poder reescalar sin problemas se debe elegir una escala inicial menor que el número expresado por el último valor de \verb|coeff_modulus| (cuando \verb|coeff_modulus| está integro). Además, el primer valor de la cadena (el que se utilizará cuando se vaya a descifrar), debe ser mayor que el resto para poder tener cierta precisión: si este valor es 60, y el segundo (el que se utiliza al realizar la última operación) es 40, tendremos 20 bits de precisión ($60 - 40$) para los decimales al retirar la escala.

  A continuación veremos en qué consiste exactamente el contexto.

\end{itemize}

\subsection{SEALContext, niveles de error y realinearización}

\verb|SEALContext| (lo que llamamos contexto, o contexto criptográfico) es la clase en la que se codifican las propiedades del texto cifrado, desde el esquema utilizado a los distintos parámetros de cifrado. La creación del \verb|SEALContext| genera una estructura similar a una cadena que guardará el estado de \verb|coeff_modulus|. De esta forma, podremos mantener la integridad al operar entre distintos textos cifrados símplemente verificando que esta cadena es igual.

Inicialmente, la cadena se inicializa con la información de \verb|coeff_modulus|, y va cambiando a medida que, por ejemplo, se aplican operaciones de reescalado. La cadena es una lista enlazada de los \verb|SEALContext| en cada momento de operación:

\begin{listing}[ht]
    \begin{minted}{console}
         special prime +---------+
                                 |
                                 v
coeff_modulus: { 50, 30, 30, 50, 50 }  +---+  Level 4
                                          |
                                          |
   coeff_modulus: { 50, 30, 30, 50 }  +---+  Level 3
                                          |
                                          |
       coeff_modulus: { 50, 30, 30 }  +---+  Level 2
                                          |
                                          |
           coeff_modulus: { 50, 30 }  +---+  Level 1
                                          |
                                          |
               coeff_modulus: { 50 }  +---+  Level 0
    \end{minted}
    \caption{Cadena de SEALContext (documentación de SEAL)}\label{fig:seal_levels}
\end{listing}

Se puede descender, pero nunca ascender. Cuanto más abajo, más rápidos son los cálculos, pero menos cálculos se pueden hacer. En ocasiones, debido al tamaño del texto plano, descender no tendrá consecuencias en cuanto al nivel de error, por lo que se podrá hacer para tener mayor eficiencia. En otros, será una condición indispensable para poder seguir operando. Es por esto que, cumpliendo los requisitos sobre el tamaño de \verb|coeff_modulus| con respecto a \verb|poly_modulus_degree|, y siguiendo las pautas que hemos comentado con respecto al primer y último elemento de la cadena, elijamos el mayor número de elementos intermedios posible.

Mientras que en BGV este contexto no tiene mucha importancia (más allá de la eficiencia) en CKKS la cosa cambia. Cuando se reescala en CKKS, se está eliminando al mismo tiempo parte de esta cadena. Por lo tanto, hay que tener en cuenta que si se ha reescalado, y se ha cambiado el contexto de cifrado de un elemento, tendremos que ajustar el contexto del resto de elementos que queramos operar con él (para que estén en la misma escala y usen los mismos parámetros criptográficos), ya sea reescalando (cuando sea necesario) o haciendo al elemento "descender" en la cadena.

Otra forma de reducir la velocidad a la que aumenta el nivel de error sin perder profundidad computacional (capacidad para realizar más operaciones) es la realinearización. Como hemos visto anteriormente, a medida que se opera, el error aumenta. La suma es casi gratuita, pero con el producto de dos elementos de tamaño $M$ y $N$, teniendo el resultado tamaño $M+N-1$, el error aumenta considerablemente. Interpretaremos este tamaño como el grado polinómico del elemento. La realinearización es una operación que permite, utilizando unas claves especiales (claves de realinearización), reducir el tamaño tras una multiplicación, reduciendo así el consumo de \verb|noise_budget| en las siguientes operaciones. Tiene un coste computacional alto en comparación con el resto de operaciones, pero permite operar de forma más eficiente, y que el elemento no crezca hasta "romperse" (o crezca de una forma más lenta).

\subsection{Encoders}

SEAL ofrece tres sistemas para codificar los datos con los que se va a trabajar:
\begin{itemize}
    \item \verb|IntegerEncoder| (para BFV)

    Codifica el número descomponiendolo en una sucesión de exponenciaciones para poder trabajar con sus subelementos de forma independiente.

    \item \verb|BatchEncoder| (para BFV)

    Crea una matriz $2x(N/2)$ con N el número de elementos codificables. Estos elementos son conocidos como \textit{slots}, y su número es igual a \verb|poly_modulus_degree|. En cada \textit{slot} se almacena un número módulo \verb|plain_modulus|. Si se codifica un vector, se introducirá tal cual, y si sólo se codifica un valor, se llenarán todas las posiciones de la matriz con ese valor.

    \begin{gather}
      a =
      \begin{pmatrix}
        a & a & \hdots & a \\
        a & a & \hdots & a
      \end{pmatrix} &
      \begin{pmatrix}
        c_1 & c_2 & c_3
      \end{pmatrix}
      =
      \begin{pmatrix}
        c_1 & c_2   & c_3   & 0     & \hdots & 0 \\
        0   & 0     & 0     & 0     & \hdots & 0
      \end{pmatrix}
    \end{gather}

    \item \verb|CCKSEncoder|

    Funciona de forma similar al \verb|BatchEncoder|, pero tiene la mitad de posiciones con respecto al mismo \verb|poly_modulus_degree|.
\end{itemize}{}

\subsection{Trabajo con vectores}

Tanto \verb|BatchEncoder| como \verb|CKKSEncoder|, interpretan todos los datos como vectores. Una de las cosas más interesantes de esta forma de codificar los datos, es que se pueden rotar las columnas y las filas. La rotación es cíclica, es decir, cuando se rota una posición a la izquierda el primer valor pasa a la última posición.

%
%     [  0,  1,  2,  3,  0, ...,  0,  0,  0,  0,  0 ]
%     [  4,  5,  6,  7,  0, ...,  0,  0,  0,  0,  0 ]
%
% Line  60 --> Encode and encrypt.
%     + Noise budget in fresh encryption: 134 bits
%
% Line  78 --> Rotate rows 3 steps left.
%     + Noise budget after rotation: 134 bits
%     + Decrypt and decode ...... Correct.
%
%     [  3,  0,  0,  0,  0, ...,  0,  0,  0,  1,  2 ]
%
%     [  7,  0,  0,  0,  0, ...,  0,  0,  4,  5,  6 ]
%
% Line  92 --> Rotate columns.
%     + Noise budget after rotation: 134 bits
%     + Decrypt and decode ...... Correct.
%
%     [  7,  0,  0,  0,  0, ...,  0,  0,  4,  5,  6 ]
%     [  3,  0,  0,  0,  0, ...,  0,  0,  0,  1,  2 ]

Por otro lado, las operaciones aritméticas se realizan entre los valores de cada posición uno a uno, no como una operación entre matrices. Por ejemplo, el producto sería:

\begin{gather}
\begin{bmatrix} A & B & C & D \end{bmatrix}
  \end{pmatrix}
  *
  \begin{pmatrix}
    \Box \\
    \Box \\
    \vdots{} \\
    \Box
  \end{pmatrix}
\end{gather}



\subsection{Evaluator}

Por último, conociendo todo lo que se puede hacer con SEAL, veremos cómo hacerlo.

Operaciones de Evaluator (tanto con número cifrados como con números en texto plano)

- negate

- add

- sub

- multiply

- square

- realinearize

- mod\_switch

- reescale

- exponentiate

- rotate

...


\section{TFHE}

En TFHE se trabaja directamente con puertas lógicas. No hay ningún parámetro adicional que seleccionar (mas allá de los relacionados con generar una clave aleatoria), tiene una API simple pero muy potente, pero para hacer cualquier cálculo hay que implementarlo desde el nivel más bajo. Cuando se cifra un dato se genera un array de bits correspondiente al dato cifrado. Será este array (codificado en la clase \verb|LweSample|) sobre el que aplicaremos los algoritmos y procedimientos que diseñaremos, como si trabajásemos con el texto plano.

Los elementos que forman TFHE son mucho menos complejos que los de SEAL, con un único sistema de cifrado (http://lab.algonics.net/slides_ac16/index-asiacrypt.html).

CONTAR: Sólo definir el parámetro $\lambda$ (ver http://lab.algonics.net/slides_ac16/index-asiacrypt.html#/7)

Están codificados en sólo unos pocos grupos:

\begin{itemize}
  \item Parámetros de cifrado (\texttt{TFheGateBootstrappingParameterSet})
  \item Claves: Privada ( (\texttt{TFheGateBootstrappingSecretKeySet})) y pública (\texttt{TFheGateBootstrappingCloudKeySet})
  \item Operaciones de cifrado y descifrado
  \item Operaciones lógicas sobre texto cifrado (\texttt{LweSample})
\end{itemize}

Además, TFHE ofrece funciones para limpiar la memoria, dejando en manos del desarrollador la posibilidad de evitar fugas de información una vez no se va a utilizar un elemento.

\subsection{API}

A continuación haremos un repaso de las principales operaciones sobre el texto cifrado incluidas en la librería TFHE.

\begin{itemize}

  \item bootsCONSTANT

  Carga una constante en \verb|value| en la variable cifrada \verb|result|.

  \begin{lstlisting}[language=c++]
    void bootsCONSTANT(LweSample* result, int value,
                      const TFheGateBootstrappingCloudKeySet* bk);
  \end{lstlisting}

  \item bootsNOT

  Niega el valor de la variable \verb|ca| y lo almacena en \verb|result|.

  \begin{lstlisting}[language=c++]
    void bootsNOT(LweSample* result, const LweSample* ca,
                  const TFheGateBootstrappingCloudKeySet* bk);
  \end{lstlisting}

  % TODO Alinear y verbatim
  % TODO Etiquetar los códigos
  \item bootsCOPY

  Copia el valor de la variable \verb|ca| y lo almacena en \verb|result|. Cuando se utilizan tanto esta función como las funciones \verb|bootsCONSTANT|  y \verb|bootsNOT| se suele trabajar bit a bit iterando sobre los dos arrays. Si no se hace así, a veces pueden tener comportamientos erráticos.

  \begin{lstlisting}[language=c++]
    bootsCOPY(LweSample* result, const LweSample* ca,
              const TFheGateBootstrappingCloudKeySet* bk);
  \end{lstlisting}

  \item bootsMUX

  Es una implementación del operador ternario (\verb|a ? b : c|) esencial para poder introducir cómputos con divergencias en su flujo de ejecución aunque (como hemos comentado anteriormente) no podamos modificarlo. Asigna a \verb|result| el valor de \verb|b| si se cumple \verb|a|, si no, le asigna el valor de \verb|c|.

  \begin{lstlisting}[language=c++]
    bootsMUX(LweSample* result, const LweSample* a,
            const LweSample* b, const LweSample* c,
            const TFheGateBootstrappingCloudKeySet* bk);
  \end{lstlisting}

  Esta función es especialmente interesante, y es la que le da todo el valor a la librería para hacer implementaciones complejas. Por ejemplo, una función con el siguiente código:

  \begin{lstlisting}[language=c++]
    while (result < 100)
      result = result * 2;
  \end{lstlisting}

  No podría ser implementada sin evaluar el valor de result. Sin embargo, con el operador \verb|MUX| podemos hacer lo siguiente (es pseudocódigo):

  \begin{lstlisting}[language=c++]
    /*
     Hasta que el menor valor que podamos
     escribir con los bits que hemos asignado a
     los decimales (10 bits) no sea mayor que 100
    */
    for (int i = 0.001; i < 100; i = i*2) {
      // es_mayor =  result >= 100
      gte(es_mayor, result, 100);
      // factor = es_mayor ? 2 : 1
      bootsMUX(factor, es_mayor, 2, 1);
      // result = result * factor
      multiplica(result, result, factor);
    }
  \end{lstlisting}

  \item Puertas lógicas

  También implementa las siguientes puertas lógicas booleanas. La utilizaremos como base para hacer el resto de operaciones, como por ejemplo la suma (la función de suma es igual que un circuito sumador).

  \begin{itemize}
    \item NAND

    \item OR

    \item AND

    \item XOR

    \item XNOR

    \item NOR

    \item ANDNY, ANDYN

    \item ORNY, ORYN

  \end{itemize}

\end{itemize}

\subsection{Evolución de TFHE}

En su desarrollo se plantea la implementación de un modo conocido como Chimera (\cite{boura_chimera:_2018}) en el que se puedan mover los datos entre el esquema de TFHE y esquemas más eficientes como BFV, CKKS; y viceversa, sin tener que descifrarlos. Por ejemplo, esto puede ser útil para trabajar con muchas puertas lógicas y hacer alguna operación aritmética rápida entre medias.

  \chapter{Solución propuesta}

Para analizar las librerías he elaborado un sistema de posicionamiento anónimo en función de la temperatura y el mes del año.

En este sistema habrá tres actores: el cliente, el servidor de posicionamiento (programado con SEAL) y un tercer servidor (programado con TFHE) que generará el modelo para calcular la posición.

El cliente consultará su posición con el servidor de SEAL, que previamente habrá generado en el servidor de TFHE un modelo de posicionamiento basado en las temperaturas del último año.

% TODO NO debe ser publico el descifrado porque es vulnerable a CCA

\section{Funcionamiento}

% TODO Gráficos de curvas de regresión con ejemplos

\subsection{Generación del modelo}

1. El cliente genera un par de claves

2. Cifra n pares de datos (en nuestro ejemplo el par sería (mes, temperatura)).

3. Sube los datos cifrados y su clave pública. El número de datos que se puede subir está limitado por el crecimiento del tamaño (en bits) de dichos datos al exponenciarlos para calcular la curva de regresión. Trabajaremos con los datos de 12 meses porque, como comentaremos más adelante, aunque el orden máximo al que llegaríamos con estos datos es de 46 bits tendremos otras limitaciones a la hora de procesar los datos.

4. EL servidor procesa los datos


\section{Implementación}

\subsection{TFHE}

El servidor de SEAL cifrará los datos de temperatura del último año en dos ubicaciones distintas y se las envía al servidor TFHE. Este procesa los datos cifrados para calcular la regresión cuadrática (también cifrada) que servirá de modelo para determinar la ubicación del usuario. Esta curva se genera con tres parámetros (a, b y c):

$ ax^2 + bx + c $

Para calcularla realizará las siguientes operaciones:

$ OPERACIONES PARA a, b y c $

TFHE sólo ofrece operadores lógicos, así que tenemos que escribir la operaciones aritméticas necesarias:

$ OPERACIONES DE FUNCTIONS.h $

\subsubsection{Tamaño máximo de los datos}

log(X, 2)*max_exponente <= (64 bits de entero - 10 bits de decimal - 1 bit de signo) = 53 bits

4 < max_exponente < 10

X < 32

\subsubsection{Problemas encontrados}

- Signo

- Floats

- Eficiencia

% TODO Documentar tiempos

- Tamaño de los datos al multiplicar...

% TODO Mostrar algunos ejemplos de codificación de números en bits
l = sum(1, n)
nb_bits > 1 + math.log(n, 2) + 10

\subsection{SEAL}

  \chapter{Resultados}
\label{chap:resultados}

% TODO Inlcuir datos de las máquinas

\section{TFHE}

Al ser una ejecución "especulativa"...

\subsection{Tiempos de ejecución}

\subsubsection{reg2}
%
% initVectores! 235885
% calcCuadrados! 15687
% calcDuplas! 35608
% calcComplejos! 75325
% dCalcC! 220161
% CalcB! 264239
% CalcA! 352023

% ---

% TODO Gráficos de drive
% op,bits,time(s)
% equal,4,0
% is_negative,4,0
% minimum,4,1
% maximum,4,1
% sum,4,1
% negativo,4,0
% resta,4,1
% multiply,4,5
% multiply_float,4,12
% mayor_igual,4,1
% shiftl,4,1
% shiftr,4,1
% u_shiftl,4,0
% u_shiftr,4,0
% porDiez,4,6
% entreDiez,4,15
% reescala-+,4,0
% reescala+-,4,0
% divide,4,15
% divide_float,4,57
% equal,8,0
% is_negative,8,0
% minimum,8,2
% maximum,8,2
% sum,8,1
% negativo,8,1
% resta,8,2
% multiply,8,12
% multiply_float,8,37
% mayor_igual,8,0
% shiftl,8,3
% shiftr,8,3
% u_shiftl,8,0
% u_shiftr,8,0
% porDiez,8,10
% entreDiez,8,33
% reescala-+,8,0
% reescala+-,8,0
% divide,8,57
% divide_float,8,211
% equal,16,1
% is_negative,16,0
% minimum,16,4
% maximum,16,3
% sum,16,2
% negativo,16,3
% resta,16,4
% multiply,16,31
% multiply_float,16,110
% mayor_igual,16,2
% shiftl,16,4
% shiftr,16,6
% u_shiftl,16,0
% u_shiftr,16,0
% porDiez,16,21
% entreDiez,16,72
% reescala-+,16,0
% reescala+-,16,0
% divide,16,211
% divide_float,16,801
% equal,32,2
% is_negative,32,0
% minimum,32,7
% maximum,32,7
% sum,32,4
% negativo,32,4
% resta,32,8
% multiply,32,93
% multiply_float,32,345
% mayor_igual,32,2
% shiftl,32,10
% shiftr,32,11
% u_shiftl,32,0
% u_shiftr,32,0
% porDiez,32,41
% entreDiez,32,142
% reescala-+,32,0
% reescala+-,32,0
% divide,32,812
% divide_float,32,3108
% equal,64,3
% is_negative,64,0
% minimum,64,14
% maximum,64,15
% sum,64,8
% negativo,64,8
% resta,64,16
% multiply,64,330
% multiply_float,64,1257
% mayor_igual,64,4
% shiftl,64,19
% shiftr,64,22
% u_shiftl,64,0
% u_shiftr,64,0
% porDiez,64,82
% entreDiez,64,294
% reescala-+,64,0
% reescala+-,64,0
% divide,64,3076
% divide_float,64,11662

% ---

% /*
%   multi = 5
%   suma = 7
%
%   t_multi = 1257
%   t_suma = 8
%
%   t_total = (t_multi*multi + t_suma*suma)*values
%           = (6341*values)s = (105.68*values) minutos
% */
% void RegresionCuadratica::initVectores(const vector<LweSample*> xs, const vector<LweSample*> ys, string results_path){
%
%
% /*
%   multi = 4
%
%   t_multi = 1257
%
%   t_total = t_multi*multi = 5028s = 83.8 minutos
% */
% void RegresionCuadratica::calcCuadrados(string results_path)
%
%
% /*
%   multi = 9
%
%   t_multi = 1257
%
%   t_total = t_multi*multi = 11313s = 188.55m = 3.14 horas
% */
% void RegresionCuadratica::calcDuplas(string results_path){
%
%
%
% /*
%   multi = 10
%
%   t_multi = 1257
%
%   t_total = t_multi*multi = 12570s = 209.5m = 3.49 horas
% */
% void RegresionCuadratica::calcComplejos(string results_path)
%
% /*
%   suma = 16
%   div = 4
%
%   t_suma = 8
%   t_div = 11662
%
%   t_total = t_div*div + t_suma*suma
%           = 46776s = 13 horas
% */
% void RegresionCuadratica::calcC(LweSample* c, string results_path){
%
%
% /*
%   suma = 4
%   multi = 2
%   div = 1
%
%   t_suma = 8
%   t_multi = 1257
%   t_div = 11662
%
%   t_total = t_div*div + t_multi*multi + t_suma*suma
%           = 14208s = 3.94 horas
% */
% void RegresionCuadratica::calcB(LweSample* b, LweSample* c, string results_path){
%
%
% /*
%   suma = 2
%   multi = 2
%   div = 1
%
%   t_suma = 8
%   t_multi = 1257
%   t_div = 11662
%
%   t_total = t_div*div + t_multi*multi + t_suma*suma
%           = 14192s = 3.94 horas
% */
% void RegresionCuadratica::calcA(LweSample* a, LweSample* b, LweSample* c, string results_path){

\subsection{Tamaño máximo de los datos}

% log(X, 2)*max_exponente <= (64 bits de entero - 10 bits de decimal - 1 bit de signo) = 53 bits

% 4 < max_exponente < 10

% X < 32

\subsection{Problemas encontrados}

- Signo

- Floats

- Eficiencia

% TODO Documentar tiempos

- Tamaño de los datos al multiplicar...

% TODO Mostrar algunos ejemplos de codificación de números en bits
l = sum(1, n)
nb_bits > 1 + math.log(n, 2) + 10

\section{SEAL}

\subsection{Tiempos de ejecución}

En casi todo 0

% op,poly_modulus_degree_bits,t
% slots with 8192 poly_modulus_degree: 4096
% encode_batch,13,0
% encode,13,0
% encrypt_batch,13,0
% encrypt_number,13,0
% multiply,13,0
% add,13,0
% slots with 16384 poly_modulus_degree: 8192
% encode_batch,14,0
% encode,14,0
% encrypt_batch,14,0
% encrypt_number,14,0
% multiply,14,0
% add,14,0
% slots with 32768 poly_modulus_degree: 16384
% encode_batch,15,0
% encode,15,0
% encrypt_batch,15,0
% encrypt_number,15,1
% multiply,15,0
% add,15,0

\subsection{Límites de cómputo}

Con CKKS, por tamaño de la cadena:

El primer y el último número de la cadena tienen que ser mayores que el número a cifrar/descifrar, y los intermedios tienen que ser lo algo más grandes  que los intermedios para asegurar la precisión, y la suma de estos dos con los intermedios tiene que ser menos que  \verb|max coeff_modulus| bit-length. POr lo tanto:

\begin{itemize}
    \item Para un número de 40 bits, es necesario utilizr al menos  \verb|poly_modulus_degree de 8192|, y se pueden hacer 4 operaciones.
    \item Para un número de 64 bits, es necesario utilizr al menos  \verb|poly_modulus_degree de 16384|, y se pueden hacer 7 operaciones.
    \item Para un número de 64 bits, es necesario utilizr al menos \verb|poly_modulus_degree de 16384|, y se pueden hacer 14 operaciones.
\end{itemize}

Este es el número de operaciones tras el cual el resultado es inválido.

Con BFV sólo enteros, y por niveles de error:

% poly_modulus_degree_bits,n
% 13,1
% 14,4
% 15,9

\section{Coste de la implantación}

Equipo de ingenieros, estudio, pruebas...

% - Estudio teórico
% - Estudio de herramientas
% - Mes con 6 horas al día

Máquina digital ocean con cálculo (tiempo*precio).

Los resultados para curva A han sido: ...
para curva B han sido: ...

  \chapter{Conclusiones}
\label{chap:conclusiones}

Aunque el sistema que hemos desarrollado, en concreto, no sirva para resolver ningún problema real (la ubicación en base a la temperatura no es un sistema fiable de posicionamiento), nuestro modelo sí que cumple con las características de un modelo real (en concreto, de machine learning) implementado con criptografía homomórfica:

\begin{itemize}
    \item Un sistema de generación de modelo lento, que sólo se tiene que ejecutar una vez
    \item Otro sistema rápido de evaluación de datos, que se ejecuta múltiples veces
\end{itemize}

Esto nos permite extraer conclusiones acerca de la viabilidad del uso de librerías de criptografía homomórfica en procesos reales:

\begin{enumerate}
    \item El problema de la eficiencia es un problema realmente grave para aplicarlo a sistemas reales: por mucho que optimizásemos nuestra solución, los $50$ ms que tarda en calcularse la curva de regresión en \textit{python} (un lenguaje que ya de por sí es lento comparado con los lenguajes compilados) equivalen a ejecutar dos puertas lógicas de TFHE.
    \item Independientemente de la eficiencia, es difícil programar soluciones con estas librerías, que funcionen, y que realmente sean seguras. Hay que saber mucho para hacerlo correctamente, y se pueden cometer errores que comprometan seriamente la seguridad (\cite{peng_danger_2019}).
\end{enumerate}

Sin embargo, existen ciertos aspectos esperanzadores en todo el asunto. Es muy buena idea en TFHE que limiten los parámetros a elegir. El desarrollador debe poder trabajar con las librerías de forma segura sin tener que elegir los parámetros o comprender qué hacen (eso debe ser trabajo de los criptógrafos).

Como hemos visto a lo largo del máster, en un ámbito profesional las medidas de seguridad tienen que implementarse en función del riesgo del sistema. Estas librerías pueden ser útiles, por ejemplo, para operaciones extremadamente confidenciales que tengan que ser ejecutadas en nubes públicas, siempre que el valor del activo así lo requiera (en comparación con el coste de desarrollar la solución, y el coste computacional extra).

En el estadío actual de las implementaciones de criptografía homomórfica no hay una solución clara que sirva para cualquier problema. Lo ideal es buscar dicha solución para cada caso concreto. Si bien esto puede incrementar el coste (es difícil reutilizar código), hace que el uso de criptografía homomórfica sea viable. Por ejemplo, mientras que nuestra solución pretende ser muy versátil para experimentar con las tecnologías, si se va a trabajar con 64 bits es mucho más eficiente implementar las operaciones en circuitos cerrados de 64 bits (como se haría en un circuito físico).

En cualquier caso, estas tecnologías prometen ser la solución a varios problemas: además de la criptografía homomórfica, la criptografía basada en \textit{lattices} es una de las aproximaciones más prometedoras en un panorama \textit{post-quantum}.

En su presentación en el año 2016, los desarrolladores de TFHE aceptaron un reto: crear sistemas FHE prácticos en menos de 10 años (figura \ref{fig:tfhe_challenge_accepted}, (\cite{chillotti_tfhe:_2016})). Los experimentos que hemos realizado con sus tecnologías, y los trabajos futuros con las tecnologías más recientes, muestran visos de optimismo.

\begin{figure}[h]
    \centering
    \includegraphics[width=\textwidth]{tfhe_challenge_accepted}
    \caption{"FHE will never be practical within the next 10 years?"}
    \label{fig:tfhe_challenge_accepted}
\end{figure}

  \chapter{Trabajos futuros}

- Estudio del nuevo estándar 4

  \appendix
  \chapter{Test LWE}
\label{appendix:test_lwe}

\section{Código en python}

\inputminted{python}{apendices/test_lwe.py}

\section{Ejecución}

\begin{minted}{console}
junquera@opa:~/tfm/lwe_tests$ python3 test_lwe.py
x > 1
c1:  [583949, 280901, 385393, 215313, 3921, 664402,
      669495, 470378, 548011, 15699]
c2:  [597668, 851152, 692041, 618895, 590806, 410594,
      694835, 830543, 381247, 582842]
Result:  [340514, 340518, 340516]
junquera@opa:~/tfm/lwe_tests$ python3 test_lwe.py
x > 0
c1:  [482569, 656029, 171078, 505801, 228565, 444569, 45811,
      394863, 509682, 357903]
c2:  [106231, 198539, 660839, 461502, 350107, 178714, 450856,
      646125, 522843, 106305]
Result:  [8, 6, 7]
\end{minted}

  \chapter{Regresión cuadrática}
\label{appendix:regresion_cuadratica}

\section{Ejemplo en python}
\inputminted{python}{apendices/regresion_cuadratica.py}

\section{Funciones en C++}

\inputminted{c++}{apendices/reg2.h}

  \chapter{Datos AEMET}
\label{appendix:datos_aemet}

\section{Obtención de datos}

Para descargar los datos de las ciudades ha sido necesario obtener una API key para acceder al sistema de Open Data (\url{https://opendata.aemet.es/centrodedescargas/inicio}, ver figura \ref{fig:panel_aemet}).

\begin{figure}[h]
  \caption{Panel de descarga de datos de AEMET}
  \label{fig:panel_aemet}
  \centering
  \includegraphics[\textwidth]{panel_aemet}
\end{figure}

Una vez descargados, hemos obtenido para cada ciudad un JSON como el siguiente:

\inputminted{json}{apendices/datos_aemet.json}

\section{Procesado}

Para cada mes, para cada archivo, he extraído el parámetro \verb|tm_mes| (que indica la temperatura media del mes) y he generado un documento CSV para poder tratarlo mejor. Los archivos generados tendrán el siguiente formato:

\inputminted{csv}{apendices/datos_aemet.csv}

  \chapter{Tests de eficiencia de SEAL}
\label{appendix:benchmarks_seal}

\section{BFV}
\inputminted{}{apendices/benchmark_bfv_seal.txt}

\section{CKKS}

\inputminted{}{apendices/benchmark_ckks_seal.txt}

  \chapter{Servidor en Digital Ocean}
\label{appendix:server-do}

\section{Servidor}

  \chapter{Manual de instalación y uso}
\label{appendix:manual}

Manual de instalación y uso de la solución del trabajo (\ref{chap:poc}). Estos pasos han sido probados en un sistema \verb|Ubuntu 19.04| de 64 bits.

\section{Prerequisitos}

\subsection{SEAL}

Como hemos comentado, Microsoft SEAL no necesita ninguna dependencia. Para instalarlo (generar las librerías y los archivos de cabeceras) seguimos los siguientes pasos:

\begin{minted}{console}
$ git clone https://github.com/Microsoft/SEAL
$ cd SEAL/native/src
$ cmake .
$ make
\end{minted}

Generará la librería compilada en la ruta \verb|SEAL/native/lib|, y tendremos sus archivos \verb|.h| en el directorio \verb|SEAL/native/src|.

Para más detalles, ver la guía de instalación oficial en \url{https://github.com/Microsoft/SEAL}.

\subsection{TFHE}

La librería TFHE sí que tiene algunos prerequisitos, que satisfaremos ejecutando:

\begin{minted}{console}
$ sudo apt-get install build-essential cmake cmake-curses-gui
\end{minted}

Una vez instalados, descargamos el código y lo configuramos para compilarlo:

\begin{minted}{console}
$ #clone the tfhe repository
$ git clone --recurse-submodules --branch=master https://github.com/tfhe/tfhe.git
$ cd tfhe
$ #configure the build options
$ mkdir build
$ cd build
$ ccmake ../src
\end{minted}

Abrirá un menú con varias opciones dependiendo de las características que queramos utilizar. En la figura \ref{fig:ccmake_tfhe} se muestran las que hemos utilizado en nuestra instalación.

\begin{figure}[h]
    \includegraphics[width=\linewidth]{ccmake_tfhe}
    \caption{Parámetros de configuración de tfhe}
    \label{fig:ccmake_tfhe}
\end{figure}

Tras guardar esta configuración, símplemente ejecutamos:

\begin{minted}{console}
$ #build the library
$ make
\end{minted}

Como resultado obtendremos la librería compilada dentro de \verb|tfhe/build/libtfhe|. En función de los parámetros que hayamos elegido tendrá un nombre u otro. Con nuestros parámetros se llamará \verb|libtfhe-fftw.so|

Para más detalles, ver la guía de instalación oficial en \url{https://tfhe.github.io/tfhe/installation.html}.

\section{Instalación}

\subsection{tfhe-math}

Para instalar \verb|tfhe-math| necesitaremos haber compilado \verb|tfhe|. Primero descargamos el código y entramos en la carpeta para configurarlo:

\begin{minted}{console}
$ git clone git@gitlab.com:junquera/tfhe-math.git
$ cd tfhe-math
\end{minted}

Editamos el archivo \verb|src/Makefile| y especificamos en la variable \verb|TFHE_PREFIX| la ruta en la que están la librería de \verb|tfhe| y sus archivos \verb|.h|. Tras guardar, entramos en la carpeta \verb|src| y ejecutamos el comando \verb|make|. Creará una carpeta llamada \verb|prefix| en la raíz del proyecto con todos los archivos necesarios para que utilicemos la librería en otros proyectos.

\subsection{tfhe-cs}

Descargaremos el código con las funcionalidades de cliente y servidor desde su repositorio en GitLab:

\begin{minted}{console}
$ git clone --recurse-submodules git@gitlab.com:junquera/tfhe-cs.git
$ cd tfhe-cs
\end{minted}

De esta forma, además se descargará la librería \verb|tfhe-math| para que la usemos (si no la hemos instalado aún).

En el archivo \verb|Makefile| tenemos que especificar la ruta a las carpetas de prefix (las carpetas que contengan las librerías y los archivos de cabeceras \verb|.h|) de \verb|tfhe| y de \verb|tfhe-math|.

Ejecutando el comando \verb|make| generaremos el programa cliente, el programa servidor y el ejecutable con los test.

\subsection{seal-cs}

Haremos exactamente lo mismo que con \verb|tfhe-cs|, pero eligiendo su repositorio correspondiente:

\begin{minted}{console}
$ git clone git@gitlab.com:junquera/seal-cs.git
$ cd tfhe-cs
\end{minted}

Esta vez, en lugar de especificar la ruta a \verb|tfhe|, indicamos en el archivo \verb|Makefile| la ruta a \verb|SEAL| y ejecutamos el comando \verb|make|.

\section{Uso}

A continuación veremos los distintos usos que podemos darle a nuestra implementación, ya sea para ejecutar el código del trabajo o realizar nuestra propia implementación. Las cabeceras de referencia de cada programa se pueden ver en el anexo \ref{appendix:cs-headers}.

\subsection{tfhe-math}

Para utilizar las funciones de \verb|tfhe-math| únicamente tenemos que enlazar nuestro código 
con la librería, e incluir sus archivos de cabecera en él. Por ejemplo, para utilizar las funciones aritméticas escribiríamos:

\begin{minted}{c++}
#include "tfhe-math/arithmetic.h"
\end{minted}

Después trabajaríamos con sus métodos, que funcionan tal y como hemos indicado en la sección \ref{tag:tfhe-math-ops}, teniendo en cuenta únicamente que:

\begin{itemize}
    \item Los métodos precedidos con \verb|u_| son más rápidos, pero no tienen en cuenta el sigo.
    \item Si se trabaja con números reales, para multiplicar y dividir usaremos las funciones \verb|multiply_float| y \verb|divide_float| respectivamente (en lugar de \verb|multiply| y \verb|divide|)
\end{itemize}

\subsection{tfhe-cs}

Si quisiésemos usar la funcionalidad del cliente o del servidor (no del programa, si no de la clase) bastaría con utilizar los códigos \verb|client.cpp| o \verb|server.cpp| (respectivamente) incluyendo sus archivos de cabeceras.

\subsection{seal-cs}

Al igual que con \verb|tfhe-cs| podemos utilizar nuestros códigos de cliente y servidor para hacer una implementación propia con sus funcionalidades, o ejecutar los programas generados (\verb|client|, \verb|server| y \verb|test|).

  \bibliographystyle{apalike}
  \nocite{*}
  \bibliography{references.bib}

\end{document}
