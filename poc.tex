\chapter{Prueba de concepto}

Para analizar las librerías he elaborado un sistema de posicionamiento anónimo en función de la temperatura y el mes del año. 

En este sistema habrá tres actores: el cliente, el servidor de posicionamiento (programado con SEAL) y un tercer servidor (programado con TFHE) que generará el modelo para calcular la posición.

El cliente consultará su posición con el servidor de SEAL, que previamente habrá generado en el servidor de TFHE un modelo de posicionamiento basado en las temperaturas del último año.

\section{TFHE}

El servidor de SEAL cifrará los datos de temperatura del último año en dos ubicaciones distintas y se las envía al servidor TFHE. Este procesa los datos cifrados para calcular la regresión cuadrática (también cifrada) que servirá de modelo para determinar la ubicación del usuario. Esta curva se genera con tres parámetros (a, b y c):

ax^2 + bx + c

Para calcularla realizará las siguientes operaciones:

$ OPERACIONES PARA a, b y c $

TFHE sólo ofrece operadores lógicos, así que tenemos que escribir la operaciones aritméticas necesarias:

$ OPERACIONES DE FUNCTIONS.h $

\subsection{Tamaño máximo de los datos}

log(X, 2)*max_exponente <= (64 bits de entero - 10 bits de decimal - 1 bit de signo) = 53 bits

4 < max_exponente < 10

X < 32

\subsection{Problemas encontrados}

- Signo

- Floats

- Eficiencia

- Tamaño de los datos al multiplicar...


% Documentar problemas
