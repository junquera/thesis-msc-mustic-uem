\chapter{Introducción}
\label{chap:intro}

Para todo $ A,B \in \chi{} $, una operación $ \bigoplus $, y una función $f$; si se cumple $ f(A) \bigoplus f(B) = f(A \bigoplus B)$, $ f $ es una función homomórfica con respecto a $ \bigoplus $ en $ \chi{} $. Cuando una función criptográfica cumple esta condición con alguna operación se dice que es maleable (\cite{dolev_non-malleable_1991}). 

Aunque es una propiedad que podría no ser deseable en muchos ámbitos (por ejemplo, en un escenario en el que además de confidencialidad se requiera integridad, permitiría a un atacante modificar los datos), si cumple ciertas condiciones podría tener numerosas aplicaciones. Así se crea el campo de estudio de la criptografía homomórfica.

Se conoce como criptografía homomórfica al conjunto de técnicas criptográficas destinadas a permitir operar con datos cifrados y que dichas operaciones se materialicen correctamente sobre los datos al descifrarlos. En función (principalmente) de las operaciones con las que se cumple esta premisa, o el sistema utilizado para procesar los datos antes y después de operar... Aunque pueda haber muchas variantes de cada una de estas propiedades, la comunidad científica ha establecido criterios y notaciones para su estudio. 

\section{Estandarización}

El consorcio "Homomorphic Encryption Standardization" (\cite{noauthor_homomorphic_nodate-1}) ha ido desarrollando un estándar a lo largo de los años atendiendo a los avances en las distintas tecnologías que componen la criptografía homomórfica, y prestando un interés especial en las implementaciones necesarias para ponerla en práctica. Así, se han ido sucediendo las tres generaciones de criptografía homomórfica.

El último encuentro de trabajo del grupo se produjo el 17 de Agosto en Santa Clara, en el que se trabajará principalmente en la eficiencia y la ayuda al desarrollador (usabilidad). http://homomorphicencryption.org/aug-17-2019-homomorphicencryption-org-standards-meeting/

\subsection{Seguridad}

En el documento \cite{chase_security_2017} ...

- Estudio de las bases seguras de cada tecnología

- Estudio de los ataques conocidos

- Estudio de los parámetros de seguridad recomendados para el escenario actual y un escenario "post-quantum" \cite{citar}

\subsection{API}

\cite{brenner_standard_2017}

(1) a ​storage model​ to identify whatneeds to be included to both serialize and deserialize keys, ciphertexts, plaintexts, encryptionparameters, and scheme/implementation dependent data, and to support homomorphiccomputations; (2) an ​assembly language​ -like representation of homomorphic encryptionprograms, consisting of low-level library calls. These two topics will form the core of thehomomorphic encryption standard.

- SecKeygen

- PubKeygen

- SecEncrypt

- Encrypt

- Decrypt

- Eval


Algunos esquemas meten realinearización, rotación...

\subsection{Aplicaciones}

\cite{archer_applications_2017}

- Médicas

- Computación distribuida

- Sistemas de control

Ver: https://homomorphicencryption.org/wp-content/uploads/2019/08/juan_troncoso-pastoriza_kurt_rohloff.pdf

\section{Objetivo del trabajo}

El objetivo de este trabajo es evaluar si es viable o no utilizar las tecnologías existentes en sistemas y procesos reales. Estudiaremos qué implementaciones hay, en qué consisten, y cuales serán las más idóneas (las más avanzadas, que puedan servir de muestra para inferir la viabilidad de las demás) atendiendo a:

\begin{itemize}
    \item La facilidad de uso: A fin de cuentas la documentación existente y la mayor o menor facilidad de uso se traduce en horas de salario de trabajadores altamente cualificados.
    \item Las capacidades de la tecnología: O qué problemas pueden resolverse con ella
    \item La eficiencia de la solución: Estudiar los tiempos de ejecución de las operaciones
\end{itemize}